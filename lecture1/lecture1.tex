\documentclass[psamsfonts, 12pt]{amsart}
%
%-------Packages---------
%
\usepackage[h margin=1 in, v margin=1 in]{geometry}
\usepackage{amssymb,amsfonts}
\usepackage[all,arc]{xy}
\usepackage{tikz-cd}
\usepackage{enumerate}
\usepackage{mathrsfs}
\usepackage{amsthm}
\usepackage{mathpazo}
\usepackage{float}
\usepackage[backend=biber]{biblatex}
\addbibresource{bibliography.bib}
%\usepackage{charter} %another font
%\usepackage{eulervm} %Vakil font
\usepackage{yfonts}
\usepackage{mathtools}
\usepackage{enumitem}
\usepackage{mathrsfs}
\usepackage{fourier-orns}
\usepackage[all]{xy}
\usepackage{hyperref}
\usepackage{url}
\usepackage{mathtools}
\usepackage{graphicx}
\usepackage{pdfsync}
\usepackage{mathdots}
\usepackage{calligra}
\usepackage{import}
\usepackage{xifthen}
\usepackage{pdfpages}
\usepackage{transparent}

\newcommand{\incfig}[2]{%
    \fontsize{48pt}{50pt}\selectfont
    \def\svgwidth{\columnwidth}
    \scalebox{#2}{\input{#1.pdf_tex}}
}
%
\usepackage{tgpagella}
\usepackage[T1]{fontenc}
%
\usepackage{listings}
\usepackage{color}

\definecolor{dkgreen}{rgb}{0,0.6,0}
\definecolor{gray}{rgb}{0.5,0.5,0.5}
\definecolor{mauve}{rgb}{0.58,0,0.82}

\lstset{frame=tb,
  language=Matlab,
  aboveskip=3mm,
  belowskip=3mm,
  showstringspaces=false,
  columns=flexible,
  basicstyle={\small\ttfamily},
  numbers=none,
  numberstyle=\tiny\color{gray},
  keywordstyle=\color{blue},
  commentstyle=\color{dkgreen},
  stringstyle=\color{mauve},
  breaklines=true,
  breakatwhitespace=true,
  tabsize=3
  }
%
%--------Theorem Environments--------
%
\newtheorem{thm}{Theorem}[section]
\newtheorem*{thm*}{Theorem}
\newtheorem{cor}[thm]{Corollary}
\newtheorem{prop}[thm]{Proposition}
\newtheorem{lem}[thm]{Lemma}
\newtheorem*{lem*}{Lemma}
\newtheorem{conj}[thm]{Conjecture}
\newtheorem{quest}[thm]{Question}
%
\theoremstyle{definition}
\newtheorem{defn}[thm]{Definition}
\newtheorem*{defn*}{Definition}
\newtheorem{defns}[thm]{Definitions}
\newtheorem{con}[thm]{Construction}
\newtheorem{exmp}[thm]{Example}
\newtheorem{exmps}[thm]{Examples}
\newtheorem{notn}[thm]{Notation}
\newtheorem{notns}[thm]{Notations}
\newtheorem{addm}[thm]{Addendum}
\newtheorem{exer}[thm]{Exercise}
%
\theoremstyle{remark}
\newtheorem{rem}[thm]{Remark}
\newtheorem*{claim}{Claim}
\newtheorem*{aside*}{Aside}
\newtheorem*{rem*}{Remark}
\newtheorem*{hint*}{Hint}
\newtheorem*{note}{Note}
\newtheorem{rems}[thm]{Remarks}
\newtheorem{warn}[thm]{Warning}
\newtheorem{sch}[thm]{Scholium}
%
%--------Macros--------
\renewcommand{\qedsymbol}{$\blacksquare$}
\renewcommand{\sl}{\mathfrak{sl}}
\newcommand{\Bord}{\mathsf{Bord}}
\renewcommand{\hom}{\mathsf{Hom}}
\renewcommand{\emptyset}{\varnothing}
\renewcommand{\O}{\mathscr{O}}
\newcommand{\R}{\mathbb{R}}
\newcommand{\ib}[1]{\textbf{\textit{#1}}}
\newcommand{\Q}{\mathbb{Q}}
\newcommand{\Z}{\mathbb{Z}}
\newcommand{\N}{\mathbb{N}}
\newcommand{\C}{\mathbb{C}}
\newcommand{\A}{\mathbb{A}}
\newcommand{\F}{\mathbb{F}}
\newcommand{\M}{\mathcal{M}}
\newcommand{\dbar}{\overline{\partial}}
\newcommand{\zbar}{\overline{z}}
\renewcommand{\S}{\mathbb{S}}
\newcommand{\V}{\vec{v}}
\newcommand{\RP}{\mathbb{RP}}
\newcommand{\CP}{\mathbb{CP}}
\newcommand{\B}{\mathcal{B}}
\newcommand{\GL}{\mathsf{GL}}
\newcommand{\SL}{\mathsf{SL}}
\newcommand{\SP}{\mathsf{SP}}
\newcommand{\SO}{\mathsf{SO}}
\newcommand{\SU}{\mathsf{SU}}
\newcommand{\gl}{\mathfrak{gl}}
\newcommand{\g}{\mathfrak{g}}
\newcommand{\Bun}{\mathsf{Bun}}
\newcommand{\inv}{^{-1}}
\newcommand{\bra}[2]{ \left[ #1, #2 \right] }
\newcommand{\set}[1]{\left\lbrace #1 \right\rbrace}
\newcommand{\abs}[1]{\left\lvert#1\right\rvert}
\newcommand{\norm}[1]{\left\lVert#1\right\rVert}
\newcommand{\transv}{\mathrel{\text{\tpitchfork}}}
\newcommand{\defeq}{\vcentcolon=}
\newcommand{\enumbreak}{\ \\ \vspace{-\baselineskip}}
\let\oldexists\exists
\renewcommand\exists{\oldexists~}
\let\oldL\L
\renewcommand\L{\mathfrak{L}}
\makeatletter
\newcommand{\tpitchfork}{%
  \vbox{
    \baselineskip\z@skip
    \lineskip-.52ex
    \lineskiplimit\maxdimen
    \m@th
    \ialign{##\crcr\hidewidth\smash{$-$}\hidewidth\crcr$\pitchfork$\crcr}
  }%
}
\makeatother
\newcommand{\bd}{\partial}
\newcommand{\lang}{\begin{picture}(5,7)
\put(1.1,2.5){\rotatebox{45}{\line(1,0){6.0}}}
\put(1.1,2.5){\rotatebox{315}{\line(1,0){6.0}}}
\end{picture}}
\newcommand{\rang}{\begin{picture}(5,7)
\put(.1,2.5){\rotatebox{135}{\line(1,0){6.0}}}
\put(.1,2.5){\rotatebox{225}{\line(1,0){6.0}}}
\end{picture}}
\DeclareMathOperator{\id}{id}
\DeclareMathOperator{\im}{Im}
\DeclareMathOperator{\codim}{codim}
\DeclareMathOperator{\coker}{coker}
\DeclareMathOperator{\supp}{supp}
\DeclareMathOperator{\inter}{Int}
\DeclareMathOperator{\sign}{sign}
\DeclareMathOperator{\sgn}{sgn}
\DeclareMathOperator{\indx}{ind}
\DeclareMathOperator{\alt}{Alt}
\DeclareMathOperator{\Aut}{Aut}
\DeclareMathOperator{\trace}{trace}
\DeclareMathOperator{\ad}{ad}
\DeclareMathOperator{\End}{End}
\DeclareMathOperator{\Ad}{Ad}
\DeclareMathOperator{\Lie}{Lie}
\DeclareMathOperator{\spn}{span}
\DeclareMathOperator{\dv}{div}
\DeclareMathOperator{\grad}{grad}
\DeclareMathOperator{\Sym}{Sym}
\DeclareMathOperator{\sheafhom}{\mathscr{H}\text{\kern -3pt {\calligra\large om}}\,}
\newcommand*\myhrulefill{%
   \leavevmode\leaders\hrule depth-2pt height 2.4pt\hfill\kern0pt}
\newcommand\niceending[1]{%
  \begin{center}%
    \LARGE \myhrulefill \hspace{0.2cm} #1 \hspace{0.2cm} \myhrulefill%
  \end{center}}
\newcommand*\sectionend{\niceending{\decofourleft\decofourright}}
\newcommand*\subsectionend{\niceending{\decosix}}
\def\upint{\mathchoice%
    {\mkern13mu\overline{\vphantom{\intop}\mkern7mu}\mkern-20mu}%
    {\mkern7mu\overline{\vphantom{\intop}\mkern7mu}\mkern-14mu}%
    {\mkern7mu\overline{\vphantom{\intop}\mkern7mu}\mkern-14mu}%
    {\mkern7mu\overline{\vphantom{\intop}\mkern7mu}\mkern-14mu}%
  \int}
\def\lowint{\mkern3mu\underline{\vphantom{\intop}\mkern7mu}\mkern-10mu\int}
%
%--------Hypersetup--------
%
\hypersetup{
    colorlinks,
    citecolor=black,
    filecolor=black,
    linkcolor=blue,
    urlcolor=blacksquare
}
%
%--------Solution--------
%
\newenvironment{solution}
  {\begin{proof}[Solution]}
  {\end{proof}}
%
%--------Graphics--------
%
%\graphicspath{ {images/} }

\begin{document}
%
\author{Kurtis David}
%
\title{CS388C: Lecture 1}
%
\maketitle
%

\section{Circuits}

\begin{defn}
\underline{Circuit} (\textbf{ckt}) - a function $f: \{0,1\}^n \rightarrow \{0,1\}$. i.e. any function of $n$ bits is computable by a \textbf{ckt} of \textit{size} $\leq n\cdot2^n$
\end{defn}

\begin{defn}
\underline{Size} (of a ckt) - \# of nodes in a ckt. e.g. and, or, not gates.
\end{defn}

\begin{quest}
Do $\exists$ functions requiring exponential size ckts?
\end{quest}

\textbf{Answer:} Yes.

\begin{solution}\
\vspace{0.5em}
\begin{itemize}
  \item Many possible $f: \{0,1\}^n\rightarrow\{0,1\}$
  \begin{itemize}
    \item $2^n$ inputs, $2$ outputs $\Rightarrow 2^{2^{n}}$ possible circuits
  \end{itemize}
  \item "Few" ckts of small size $s$ (few in comparison to above)
\end{itemize}

\[
\text{\# ckts of size } s \leq \text{(\# possible ways to add a new gate)}^s
\]

But there are 3 types of nodes (add/or/not) and $s\choose2$ nodes to choose from when adding these gates so \# possible ways = $\frac{3s(s-1)}{2}$.

\[
\therefore \text{\# ckts of size } s \leq (2s^2)^s \leq s^{3s}
\]

If we say $s = \frac{2^n}{3n}$ then we end up with $... \leq \frac{2^n}{3n}^{\frac{2^n}{n}} = \frac{1}{3n}^{\frac{2^n}{n}} {2^n}^{\frac{2^n}{n}}$

$\therefore$ most fns require ckts of size $\geq \frac{2^n}{3n}$

\end{solution}

\section{Languages}

\begin{defn}
\underline{Language} - a subset of $\Sigma^{*} = \bigcup_{i=0}^{\infty}\Sigma^{i}$ mapped to $\Sigma = \{0,1\}$. i.e. $f:\Sigma^{*} \rightarrow \Sigma $ (all lengths of binary strings!)
\end{defn}

To show $NP\neq P$, it suffices to show a language in $NP$ requires ckt families of size $> n^c\ \forall c$.

\underline{Open Q}: Find a language in $NP$ requiring ckts of size $\omega(n)$

\pagebreak

\section{Basic Counting}

\begin{quest}
How many ways can $r$ objects be placed in $n$ bins?
\end{quest}

\begin{solution}\
\vspace{0.5em}
Assume bins are distinguishable.

\textbf{1a. Let objects be distinguishable.}

$\Rightarrow$ $n^r$ ways (since each object can be placed in any of the $n$ bins).

\vspace{0.5em}
\textbf{1b. Now assume 1a AND $\leq 1$ object/bin.}

$\Rightarrow$ $n$P$r = \frac{n!}{(n-r)!}$. This is because after placing an object into some distinguishable bin, you can no longer choose this bin again.

(so the product would look like $n\cdot(n-1)\cdot(n-2)\cdot ... $)

\vspace{1em}

\textbf{2a. Let objects be indistinguishable.}

\vspace{0.5em}

\textbf{2b. If $\leq 1$ object/bin, simple.} $n\choose r$, since we have repeated permutations.

\vspace{0.5em}
For 2a, we must be more careful. One "trick" is to use a sticks argument -- suppose all $r$ objects are layed across a floor -- to create bins, simply add \textit{boundaries} before or after each object.

To have $n$ bins, we will have $n-1$ boundaries. There exist at least $r$ spaces, but because we can put all objects into the first bin, there must be at least $n-1$ spaces at the end to put all the boundaries at.

$\therefore$ there are ${r+n-1}\choose{n-1}$.
\end{solution}

\begin{rem*}
We only covered 4 situations, there is something called the \textbf{Twelvefold way} that includes when bins are indistinguishable, and if each bin has $\geq 1$ object each.
\end{rem*}

\begin{quest}
How many triples $(a,b,c)$ of nonnegative integers are there $s.t.\ a+b+c=100$?
\end{quest}

\begin{solution}

This question can be reformulated as the previous, where we have $r=100$ indistinguishable objects and $n=3$ bins. Thus we have $102\choose2$ ways.
\end{solution}

\begin{quest}
Suppose you can place $n$ points on a circle s.t. no 3 chords between them meet at an interior point. How many interior intersection points are there?
\end{quest}

\begin{solution} \
\vspace{0.5em}

Notice that an intersection point can be determine by 2 lines i.e. 4 points.

$\therefore$ in terms of $n$, we have $n\choose4$
\end{solution}

\begin{quest}
Show that \# people who shake hands an odd \# of times is even.
\end{quest}

\begin{solution}
\ \vspace{0.5em}

Create a handshake graph -- let nodes be people, and edges represent a handshake between them.

Thus \# handshakes $= \Sigma \text{deg}(v) = 2*edges$

But also $\#$ handshakes $= \Sigma_{\text{deg}(v) \text{ odd}} \text{deg}(v) + \Sigma_{\text{deg}(v) \text{ even}} \text{deg}(v)$
\end{solution}

\section{Hypergraphs}
\begin{defn}
\underline{Hypergraph} - graph that contains (hyper)edges can contain an arbitrary \# of nodes (instead of just 2)
\end{defn}

\begin{defn}
\underline{k-uniform hypergraph} - each hyperedge contains $k$ nodes.

$deg(v) = $ \# hyperedges containing $v$

$\Rightarrow k | \Sigma\text{deg}(v)$
\end{defn}
\end{document}
