\documentclass[psamsfonts, 12pt]{amsart}
%
%-------Packages---------
%
\usepackage[h margin=1 in, v margin=1 in]{geometry}
\usepackage{amssymb,amsfonts}
\usepackage[all,arc]{xy}
\usepackage{tikz-cd}
\usepackage{enumerate}
\usepackage{mathrsfs}
\usepackage{amsthm}
\usepackage{mathpazo}
\usepackage{float}
\usepackage[backend=biber]{biblatex}
\addbibresource{bibliography.bib}
%\usepackage{charter} %another font
%\usepackage{eulervm} %Vakil font
\usepackage{yfonts}
\usepackage{mathtools}
\usepackage{enumitem}
\usepackage{mathrsfs}
\usepackage{fourier-orns}
\usepackage[all]{xy}
\usepackage{hyperref}
\usepackage{url}
\usepackage{mathtools}
\usepackage{graphicx}
\usepackage{pdfsync}
\usepackage{mathdots}
\usepackage{calligra}
\usepackage{import}
\usepackage{xifthen}
\usepackage{pdfpages}
\usepackage{transparent}

\newcommand{\incfig}[2]{%
    \fontsize{48pt}{50pt}\selectfont
    \def\svgwidth{\columnwidth}
    \scalebox{#2}{\input{#1.pdf_tex}}
}
%
\usepackage{tgpagella}
\usepackage[T1]{fontenc}
%
\usepackage{listings}
\usepackage{color}

\definecolor{dkgreen}{rgb}{0,0.6,0}
\definecolor{gray}{rgb}{0.5,0.5,0.5}
\definecolor{mauve}{rgb}{0.58,0,0.82}

\lstset{frame=tb,
  language=Matlab,
  aboveskip=3mm,
  belowskip=3mm,
  showstringspaces=false,
  columns=flexible,
  basicstyle={\small\ttfamily},
  numbers=none,
  numberstyle=\tiny\color{gray},
  keywordstyle=\color{blue},
  commentstyle=\color{dkgreen},
  stringstyle=\color{mauve},
  breaklines=true,
  breakatwhitespace=true,
  tabsize=3
  }
%
%--------Theorem Environments--------
%
\newtheorem{thm}{Theorem}[section]
\newtheorem*{thm*}{Theorem}
\newtheorem{cor}[thm]{Corollary}
\newtheorem{prop}[thm]{Proposition}
\newtheorem{lem}[thm]{Lemma}
\newtheorem*{lem*}{Lemma}
\newtheorem{conj}[thm]{Conjecture}
\newtheorem{quest}[thm]{Question}
%
\theoremstyle{definition}
\newtheorem{defn}[thm]{Definition}
\newtheorem*{defn*}{Definition}
\newtheorem{defns}[thm]{Definitions}
\newtheorem{con}[thm]{Construction}
\newtheorem{exmp}[thm]{Example}
\newtheorem{exmps}[thm]{Examples}
\newtheorem{notn}[thm]{Notation}
\newtheorem{notns}[thm]{Notations}
\newtheorem{addm}[thm]{Addendum}
\newtheorem{exer}[thm]{Exercise}
%
\theoremstyle{remark}
\newtheorem{rem}[thm]{Remark}
\newtheorem*{claim}{Claim}
\newtheorem*{aside*}{Aside}
\newtheorem*{rem*}{Remark}
\newtheorem*{hint*}{Hint}
\newtheorem*{note}{Note}
\newtheorem{rems}[thm]{Remarks}
\newtheorem{warn}[thm]{Warning}
\newtheorem{sch}[thm]{Scholium}
%
%--------Macros--------
\newcommand{\rpm}{\sbox0{$1$}\sbox2{$\scriptstyle\pm$}
  \raise\dimexpr(\ht0-\ht2)/2\relax\box2 }
\renewcommand{\qedsymbol}{$\blacksquare$}
\renewcommand{\sl}{\mathfrak{sl}}
\newcommand{\Bord}{\mathsf{Bord}}
\renewcommand{\hom}{\mathsf{Hom}}
\renewcommand{\emptyset}{\varnothing}
\renewcommand{\O}{\mathscr{O}}
\newcommand{\R}{\mathbb{R}}
\newcommand{\ib}[1]{\textbf{\textit{#1}}}
\newcommand{\Q}{\mathbb{Q}}
\newcommand{\Z}{\mathbb{Z}}
\newcommand{\N}{\mathbb{N}}
\newcommand{\C}{\mathbb{C}}
\newcommand{\A}{\mathbb{A}}
\newcommand{\F}{\mathbb{F}}
\newcommand{\M}{\mathcal{M}}
\newcommand{\dbar}{\overline{\partial}}
\newcommand{\zbar}{\overline{z}}
\renewcommand{\S}{\mathbb{S}}
\newcommand{\V}{\vec{v}}
\newcommand{\RP}{\mathbb{RP}}
\newcommand{\CP}{\mathbb{CP}}
\newcommand{\B}{\mathcal{B}}
\newcommand{\GL}{\mathsf{GL}}
\newcommand{\SL}{\mathsf{SL}}
\newcommand{\SP}{\mathsf{SP}}
\newcommand{\SO}{\mathsf{SO}}
\newcommand{\SU}{\mathsf{SU}}
\newcommand{\gl}{\mathfrak{gl}}
\newcommand{\g}{\mathfrak{g}}
\newcommand{\Bun}{\mathsf{Bun}}
\newcommand{\inv}{^{-1}}
\newcommand{\bra}[2]{ \left[ #1, #2 \right] }
\newcommand{\set}[1]{\left\lbrace #1 \right\rbrace}
\newcommand{\abs}[1]{\left\lvert#1\right\rvert}
\newcommand{\norm}[1]{\left\lVert#1\right\rVert}
\newcommand{\transv}{\mathrel{\text{\tpitchfork}}}
\newcommand{\defeq}{\vcentcolon=}
\newcommand{\enumbreak}{\ \\ \vspace{-\baselineskip}}
\let\oldexists\exists
\renewcommand\exists{\oldexists~}
\let\oldL\L
\renewcommand\L{\mathfrak{L}}
\makeatletter
\newcommand{\tpitchfork}{%
  \vbox{
    \baselineskip\z@skip
    \lineskip-.52ex
    \lineskiplimit\maxdimen
    \m@th
    \ialign{##\crcr\hidewidth\smash{$-$}\hidewidth\crcr$\pitchfork$\crcr}
  }%
}
\makeatother
\newcommand{\bd}{\partial}
\newcommand{\lang}{\begin{picture}(5,7)
\put(1.1,2.5){\rotatebox{45}{\line(1,0){6.0}}}
\put(1.1,2.5){\rotatebox{315}{\line(1,0){6.0}}}
\end{picture}}
\newcommand{\rang}{\begin{picture}(5,7)
\put(.1,2.5){\rotatebox{135}{\line(1,0){6.0}}}
\put(.1,2.5){\rotatebox{225}{\line(1,0){6.0}}}
\end{picture}}
\DeclareMathOperator{\id}{id}
\DeclareMathOperator{\im}{Im}
\DeclareMathOperator{\codim}{codim}
\DeclareMathOperator{\coker}{coker}
\DeclareMathOperator{\supp}{supp}
\DeclareMathOperator{\inter}{Int}
\DeclareMathOperator{\sign}{sign}
\DeclareMathOperator{\sgn}{sgn}
\DeclareMathOperator{\indx}{ind}
\DeclareMathOperator{\alt}{Alt}
\DeclareMathOperator{\Aut}{Aut}
\DeclareMathOperator{\trace}{trace}
\DeclareMathOperator{\ad}{ad}
\DeclareMathOperator{\End}{End}
\DeclareMathOperator{\Ad}{Ad}
\DeclareMathOperator{\Lie}{Lie}
\DeclareMathOperator{\spn}{span}
\DeclareMathOperator{\dv}{div}
\DeclareMathOperator{\grad}{grad}
\DeclareMathOperator{\Sym}{Sym}
\DeclareMathOperator{\sheafhom}{\mathscr{H}\text{\kern -3pt {\calligra\large om}}\,}
\newcommand*\myhrulefill{%
   \leavevmode\leaders\hrule depth-2pt height 2.4pt\hfill\kern0pt}
\newcommand\niceending[1]{%
  \begin{center}%
    \LARGE \myhrulefill \hspace{0.2cm} #1 \hspace{0.2cm} \myhrulefill%
  \end{center}}
\newcommand*\sectionend{\niceending{\decofourleft\decofourright}}
\newcommand*\subsectionend{\niceending{\decosix}}
\def\upint{\mathchoice%
    {\mkern13mu\overline{\vphantom{\intop}\mkern7mu}\mkern-20mu}%
    {\mkern7mu\overline{\vphantom{\intop}\mkern7mu}\mkern-14mu}%
    {\mkern7mu\overline{\vphantom{\intop}\mkern7mu}\mkern-14mu}%
    {\mkern7mu\overline{\vphantom{\intop}\mkern7mu}\mkern-14mu}%
  \int}
\def\lowint{\mkern3mu\underline{\vphantom{\intop}\mkern7mu}\mkern-10mu\int}
%
%--------Hypersetup--------
%
\hypersetup{
    colorlinks,
    citecolor=black,
    filecolor=black,
    linkcolor=blue,
    urlcolor=blacksquare
}
%
%--------Solution--------
%
\newenvironment{solution}
  {\begin{proof}[Solution]}
  {\end{proof}}
%
%--------Graphics--------
%
%\graphicspath{ {images/} }

\begin{document}
%
\author{Kurtis David}
%
\title{CS388C: Lecture 2}
%
\maketitle
%

\section{Inclusion-Exclusion}

\begin{equation}
| A \cup B | = |A| + |B| - |A\cap B|
\end{equation}

\vspace{0.5em}
\begin{prop}
\[
|\bigcup_{i=1}^nA_i| \geq \sum_{i=1}^n|A_i| - \sum_{i<j}|A_i \cap A_j |
\]
\end{prop}

\begin{proof}
\ \\

Fix $x\in \bigcup_{i=1}^nA_i$. Say $x$ appears in \textbf{r} of the $A_i$.

Then $x$ contributes $r-{r\choose2}$ elements to the RHS. Which is equivalent to:

\begin{align*}
&= r - \frac{r(r-1)}{2}\\
&=r(1-\frac{r-1}{2})\\
&\leq 1
\end{align*}

Therefore the LHS must be greater. (See $r \geq 3$)
\end{proof}

\begin{exmp}
Suppose we have $n$ sets $A_1,...,A_n$ s.t. $|A_i| = k$ and $|A_i\cap A_j| \leq 1$.

Using \textbf{Proposition 1.1}, we have:
\[
|\cup A_i| \geq nk - {n\choose 2}
\]

However, if $k <<< n$, this can be a negative lower bound, so we can relax to only the \textbf{first k} sets...

\[
|\cup^n A_i| \geq |\cup^k A_i| \geq k^2 - {k\choose2}
\]
\end{exmp}

\begin{prop}
\begin{align*}
|\bigcup_{i=1}^nA_i| = \sum_{i=1}^n|A_i| &- \sum_{i<j}|A_i \cap A_j |\\
&+\sum_{i<j<k}|A_i \cap A_j \cap A_k |\\
&-\sum_{i<j<k<l}|A_i \cap A_j \cap A_k \cap A_l|\\
&\ \vdots\\
&\rpm |A_1 \cap A_2 \cap ... \cap A_n |
\end{align*}
\end{prop}

\begin{proof} Let use the same strategy as before. i.e. Suppose $x\in \bigcup A_i$.

Say $x$ appears in \textbf{r} of the $A_i$. We just need to find its contribution to the RHS:

\begin{align*}
\sum_{i=1}^n|A_i| &- \sum_{i<j}|A_i \cap A_j |\\
&+\sum_{i<j<k}|A_i \cap A_j \cap A_k |\\
&\ \vdots\\
&\rpm |A_1 \cap A_2 \cap ... \cap A_n | \\
&= r - {r\choose2} + {r\choose3} - {r\choose4} + ... \rpm 1\\
&= {r\choose1} - {r\choose2} + {r\choose3} - {r\choose4} + ... \rpm {r\choose r}
\end{align*}

By the binomial theorem, we know this is exactly equal to
\[
{r\choose0} - (1-1)^r = 1
\]

\end{proof}

\begin{defn}
\underline{derangement} - permutation $\pi\ s.t.\ \forall i\ \pi(i)\neq i$
\end{defn}

\begin{exmp}
Now let us try and find the \# derangements of a sequence of size $n$.
\\

We can try and frame this problem by breaking it up into smaller sets $A_i$ as before...
\\

Let $A_i = \left\lbrace \pi : \pi(i) = i \right\rbrace$. Then notice that:

\[
n! - \text{\# derangements} = |\bigcup A_i|
\]

Since there are $n!$ possible permutations. Now we can use \textbf{Proposition 1.3}. Notice that due to the definition of $A_i$, we can directly compute the value of each term -- since each set defines permutations that hold $1,2,...,n$ elements constant:

\begin{align*}
|\bigcup A_i| &= n \cdot (n-1)! - {n\choose2}\cdot(n-2)! + {n\choose3}\cdot(n-3)! + ...\ \rpm {n\choose n}1!\\
&= n! - \frac{n!}{2!(n-2)!}\cdot(n-2)! + \frac{n!}{3!(n-3)!}\cdot(n-3)! + ...\ \rpm 1\\
&= n! \cdot (1 - \frac{1}{2!} + \frac{1}{3!} + ...\ \rpm \frac{1}{n!} ) \\
&= \frac{n!}{e}
\end{align*}

Recall that
\[
exp(x) = 1 + \frac{x}{1!} + \frac{x^2}{2!} + ...
\]
\\

Thus we have that the number of derangements given $n$ is about

\[
n! (1 - \frac{1}{e})
\]
\end{exmp}

\begin{rem*}

There exist \textbf{\underline{Bonferroni Inequalities}} in the same form as \textbf{Proposition 1.1} that continuously add in longer intersection terms... e.g.:

\[
|\bigcup_{i=1}^nA_i| \leq \sum_{i=1}^n|A_i| - \sum_{i<j}|A_i \cap A_j | + \sum_{i<j<k}|A_i\cap A_j\cap A_k|
\]

\[
|\bigcup_{i=1}^nA_i| \geq \sum_{i=1}^n|A_i| - \sum_{i<j}|A_i \cap A_j | + \sum_{i<j<k}|A_i\cap A_j\cap A_k| - \sum_{i<j<k<l}|A_i\cap A_j\cap A_k\cap A_l|
\]
\end{rem*}

\section{Counting with Recursion}

\begin{exmp}
Consider a $2 \times n$ grid. $n = 5$ as shown:
\[\begin{tikzcd}
\draw[step=1.0,black,thin] grid (5,2);
\end{tikzcd}\]

\textbf{Question:} How many ways can we tile $1\times 2$ dominoes on this grid?
\ \\

\textbf{Solution:} This can be done by first noticing the following: if we place a single domino all the way to the left of this grid, we get exactly \textbf{2 situations}. The first is:

\[\begin{tikzcd}
\draw[step=1.0,black,thick, xshift=0.5cm,yshift=1cm] grid (1,2);
\draw[step=1.0,black,thin, xshift=0.5cm,yshift=1cm] grid (5,2);
\end{tikzcd}\]

Once this domino is set, notice that we now have to fill a $2\times$\textbf{(n-1)} grid. i.e. if we define $T(n) = $ \# ways to tile an $2\times n$ grid, then we would only have to find $T(n-1)$.
\\

The second situation is:
\[\begin{tikzcd}
\draw[step=1.0,black,thick, xshift=0.5cm,yshift=1cm] grid (2,1);
\draw[step=1.0,black,thin, xshift=0.5cm,yshift=1cm] grid (5,2);
\end{tikzcd}\]

When we place the domino like so, we are forced to place at least one domino to fill the space above the current one. We are then left with a $2\times $\textbf{(n-2)} grid to tile, i.e. $T(n-2)$ possibilities.
\\

Therefore, we have $T(n) = T(n-1) + T(n-2)$ as our recurrence relation.

The base cases are $T(0) = 1$ and $T(1) = 1$. The answer then leads us to the Fibonacci numbers as they share the same relation, except we start with 0 indexing. Thus we have:

\[
T(n) = F_{n+1} = (n+1)^{th} \text{ Fibonacci number }
\]

\end{exmp}

\section{Probabilistic Method}

This section will now cover examples of how to use the \textit{Probabilistic Method} to prove the existence of solutions. In general, we will be showing that there is a non-zero probability of obtaining a valid solution using some type of random assignment.

\begin{exmp}
Let $S$ be a set of elements. Let $A_i \subseteq S$ such that $|A_i|=k$. (k-set)
\\
Define family $\mathcal{F}$ as a set of k-sets:
\[
\mathcal{F} = \left\lbrace A_1,A_2,...,A_n \right\rbrace
\]
\end{exmp}

\begin{defn}
A \underline{2-coloring} of $\mathcal{F}$ is a map $\mathcal{X}: S\rightarrow \left\lbrace 0,1 \right\rbrace$ such that no $A_i$ is \textbf{monochromatic}. i.e.:
\[
\forall i \exists j,k \in A_i\ s.t.\ \mathcal{X}(j) \neq \mathcal{X}(k)
\]
\end{defn}

\begin{thm}
If $n < 2^{k-1}$, then $\exists$ 2-coloring of $\mathcal{F}$.
\end{thm}

\begin{solution}
We can employ the Probabilistic Method.
\vspace{0.5em}

First choose a coloring uniformly at random. We want to show:

\begin{center}
\textbf{With probability > 0}, we can get a valid 2-coloring of $\mathcal{F}$.
\end{center}

To do this, we can instead show its inverse:

\begin{center}
Pr$\left[ \exists \text{ monochromatic } A_i \right] < 1$
\end{center}

Let $E_i$ be the events (i.e. random colorings) such that $A_i$ is monochromatic. Then we are trying to compute:
\[
\text{Pr}\left[ \bigcup E_i \right] \leq \sum \text{Pr}\left[ E_i \right]
\]

By \textbf{Union Bound} inequality. Clearly $\text{Pr}\left[ E_i \right] = 2(\frac{1}{2})^k$, since we have 2 colors and each element is assigned with $\frac{1}{2}$ probability. Now since we know $n < 2^{k-1}$, then:

\[
\text{Pr}\left[ \bigcup E_i \right] \leq n \cdot 2(\frac{1}{2})^k < 1
\]
\end{solution}

\vspace{1em}

\begin{defn}
\underline{Universal Set}. Let us define the following notation:

\begin{center}
Set of strings of length $n$: $A \subseteq \left\lbrace 0,1 \right\rbrace^n$
\end{center}

\begin{center}
Set of $k < n$ indeces: $S = \left\lbrace i_1,i_2,...,i_k \right\rbrace$
\end{center}


\begin{center}
Restriction of A onto S: $A_{|S} = \left\lbrace (a_{i_1},a_{i_2},...,a_{i_k}) : a = (a_1,...,a_n) \in A \right\rbrace$
\end{center}

Then $A$ is a \underline{(n,k) - Universal Set} if:

\begin{center}
$\forall S$, $A_{|S}$ contains all $2^k$ possible binary strings of length $k$.
\end{center}
\end{defn}

\begin{exmp}
\textbf{Question:} How large is a smallest $(n,k)$-universal set?

\begin{thm}
$\exists\ (n,k)$-universal sets of size $\leq 2^k(k\text{ln}(n) + 1)$
\end{thm}

\begin{proof}
We will employ the Probabilistic Method again. Suppose $A$ contains $r$ random substrings. If we are trying to prove that these $(n,k)$ - universal sets of this size exist, then we want to show:

\[
\text{Pr}\left[ \exists A_{|S} \neq \left\lbrace 0,1 \right\rbrace^k \right] < 1
\]

To do this, let us further break this event into smaller ones, by considering every possible $v \in \left\lbrace 0,1 \right\rbrace^k$:

\[
\text{Pr}\left[ \bigcup_{v} \left\lbrace v\not\in A_{|S} \right\rbrace \right] \leq 2^k \cdot (1 - (\frac{1}{2})^k)^r
\]

This is because:
\begin{enumerate}
  \item Union Bound to get inequality.
  \item $2^k$ possible $v$ exist
  \item Given a substring of length $k$, $v$, then the probability of $A_{|S}$ not containing it would be $(1 - (\frac{1}{2})^k)^r$. Since $A$ contains $r$ random substrings and the probability of creating a random substring of length $k$ is $(\frac{1}{2})^k$.
\end{enumerate}

Further, we can extend the inequality:

\begin{align*}
&\leq 2^k \cdot (1 - (\frac{1}{2})^k)^r \\
&\leq {n\choose k}2^k \cdot (1-2^{-k})^r \\
&\leq \frac{n}{k!}2^k \cdot e^{-2^{-k}\cdot r}
\end{align*}

Now just set $r = 2^k(k\text{ln}(n) + 1) $. We get:

\begin{align*}
\text{Pr}\left[ \exists A_{|S} \neq \left\lbrace 0,1 \right\rbrace^k \right] &\leq
\frac{n^k}{k!}2^ke^{-k\text{ln}(n)+1}\\
&=\frac{2^k}{k!}\cdot \frac{1}{e}\\
&< 1
\end{align*}
\end{proof}
\end{exmp}

\end{document}
